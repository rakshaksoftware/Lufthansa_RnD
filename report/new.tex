\documentclass[a4paper, 12pt]{report}

%%%%%%%%%%%%
% Packages %
%%%%%%%%%%%%

\usepackage[english]{babel}
\usepackage{packages/sleek}
\usepackage{packages/sleek-title}
\usepackage{packages/sleek-theorems}
\usepackage{packages/sleek-listings}

%%%%%%%%%%%%%%
% Title-page %
%%%%%%%%%%%%%%

\logo{./resources/rakshaklogo.png}
\institute{{\bfseries\Huge Team Rakshak}\\\vspace{10pt}Indian Institute of Technology Bombay}
\faculty{Supervisor: Prof. Krishnendu Haldar\\Student Team Head: Vraj Patel}
%\department{Department of Anything but Psychology}
\title{Report: InnovAero Competition}
\subtitle{Lufthansa Technik}    
\author{\textit{List of members}\\
        Harshil Solanki - 3 Semesters BTech\\
        Shivam Chaubey - 3 Semesters BTech\\
        Jugal Shah - 5 Semesters BTech\\
        Advait Sivakumar - 5 Semesters BTech\\
        Shruti Ghoniya - 5 Semesters BTech
}
%\supervisor{Linus \textsc{Torvalds}}
% \context{Hi Hello Bye}
\date{\today}

%%%%%%%%%%%%%%%%
% Bibliography %
%%%%%%%%%%%%%%%%

\addbibresource{./resources/bib/references.bib}

%%%%%%%%%%
% Macros %
%%%%%%%%%%

\def\tbs{\textbackslash}

%%%%%%%%%%%%
% Document %
%%%%%%%%%%%%

\begin{document}
\maketitle
\romantableofcontents

\chapter{Abstract}
This report presents the results of the Lufthansa Technik InnovAero competition. The focus is on the study of contrail formation and methods to mitigate their environmental impact.

%%%%%%%%%%%%%%%%%%%%%%%%%%%%%%%%%%%%%%%%%%%%%%%%%%%%%
% Everything from here must be included in 15 pages %
%%%%%%%%%%%%%%%%%%%%%%%%%%%%%%%%%%%%%%%%%%%%%%%%%%%%%

\chapter{Introduction}
Contrails (short for "condensation trails") are line-shaped clouds produced by aircraft engine exhaust or changes in air pressure, typically at aircraft cruising altitudes several kilometres/miles above the Earth's surface \cite{enwiki:1250066651}. Contrails trap longwave radiation, contributing to net positive radiative forcing. Persistent contrails can evolve into cirrus-like clouds, which enhance warming effects by trapping heat that would otherwise escape into space. Studies show that this radiative forcing from contrails may rival or exceed $CO_2$ emissions from aviation in the short term \cite{acp}. Hence, studying contrail formation and finding ways to avoid them is crucial for reducing the environmental impact of aviation.

This study relies on the theoretically established \textbf{Schmidt-Appleman Criterion} \cite{appleman1953formation} to predict the formation of contrails and the length associated with it. Basic assumptions made here in deriving this criterion are (1) contrails are composed of ice crystals; (2) water vapor cannot be transformed into ice without first passing through the liquid phase, thus necessitating an intermediate state of saturation with respect to water. We will refer to some other studies and articles that propose the necessary equations used in thermodynamic calculations and argue upon their accuracy.

\chapter{Aircraft Components}
\section{Introduction}
Lidar polarization sensors offer a powerful tool to study the growth and evolution of contrails. By analyzing the polarization properties of the backscattered light, researchers can glean valuable insights into the microphysical properties of contrail particles, their spatial distribution, and the dynamics of their growth.

\section{Key Properties Measured by Lidar Polarization Sensors}
\subsection{Depolarization Ratio (DR)}
The ratio of the intensity of the cross-polarized component of the backscattered light to the co-polarized component.
\begin{itemize}
    \item \textbf{Significance for Contrails:} A higher DR indicates a higher degree of non-sphericity in the scattering particles. In the case of contrails, this can be indicative of the presence of non-spherical ice crystals, which can significantly affect the radiative properties of the contrail.
    \item \textbf{Growth Monitoring:} By tracking changes in DR over time, we can monitor the evolution of ice crystal shapes within the contrail. As the contrail ages, ice crystals may grow and become more complex, leading to changes in DR.
\end{itemize}

\subsection{Backscatter Coefficient}
It is a measure of the scattering intensity of atmospheric particles. A higher backscatter coefficient indicates a higher concentration of particles. For contrails, this can be used to assess the overall density of the contrail and its potential to spread and persist. By tracking changes in the backscatter coefficient over time, we can monitor the growth of the contrail's volume and mass.

\subsection{Linear Depolarization Ratio (LDR)}
It is similar to DR, but specifically measures the depolarization caused by linear scattering.
\begin{itemize}
    \item LDR can provide additional information about the orientation of ice crystals within the contrail. By analyzing the LDR, we can infer the degree of alignment of the ice crystals, which can impact the contrail's radiative properties.
\end{itemize}

\section{Sensor Placement}
Here, we propose two possible placements of a Polarization Lidar Sensor, each with some considerations.
\subsection{Tail-Mounted Placement}
\begin{itemize}
    \item \textbf{Advantages:} A tail-mounted sensor would have an excellent view of the contrail formed by the aircraft itself, enabling direct observation of the initial contrail and its evolution. This positioning allows for analysis of particle formation and behavior immediately behind the aircraft.
    \item \textbf{Challenges:} This placement may expose sensors to engine emissions or exhaust turbulence, potentially interfering with polarization readings. Additionally, structural modifications may be needed to mount sensors securely on the tail.
\end{itemize}

\subsection{Wing-Mounted Placement}
\begin{itemize}
    \item \textbf{Advantages:} Sensors can be mounted on the leading edge of a wing or on wing pylons away from the fuselage, which allows them to be clear of the engine exhaust and gives a wide view. This placement can be beneficial for scanning both upward and downward angles, enabling both atmospheric and ground-sensing capabilities.
    \item \textbf{Challenges:} Potential turbulence and vibrations from the wing might affect measurements. Additionally, interference from the wings' wake could impact data quality.
\end{itemize}






\section{Existing Sensors}
We're considering the  
\chapter{Component Installation}
% TODO: Description and illustration of how the components can be installed in the aircraft

\chapter{Calculation of Possible Contrail Formation}
\section{Theory}
The theoretical foundations are based upon the widely used Schmidt-Appleman Criterion \cite{appleman1953formation}, so much that it has become standard in most studies related to contrails. There are studies that are running experiments to verify these theoretical claims \cite{ghedhaifi2019influence}.

\chapter{Route Replanning}
% TODO: Demonstration of flight route re-planning to avoid contrails

\chapter{Conclusion}
% TODO: Conclusion and recommendations for further investigations

%%%%%%%%%%%%%%%%%%%%%%%%%%%%%%%%%%%%%%%%%%%%%%%%%%%%%%%%

% Bibliography does not fall within the 15 page limit
\printbibliography

\end{document}